%%%%%%%%%%%%%%%%%%%%%%%%%%%%%%%%%%%%%%%%%%%%%%%%%%%%%%%%%%%%%%%%%%%%%%%%
%%%%%%%%%%%%%%%%%%%%%% Simple LaTeX CV Template %%%%%%%%%%%%%%%%%%%%%%%%
%%%%%%%%%%%%%%%%%%%%%%%%%%%%%%%%%%%%%%%%%%%%%%%%%%%%%%%%%%%%%%%%%%%%%%%%
% The  template is obtained from www.tedpavlic.com
% Thanks to Ted
%%%%%%%%%%%%%%%%%%%%%%%%%%%% Document Setup %%%%%%%%%%%%%%%%%%%%%%%%%%%%

% Don't like 10pt? Try 11pt or 12pt
\documentclass[10pt]{article}
\RequirePackage[T1]{fontenc}

% To convert to times use the follow
%usepackage{times}  % <<=======
%\renewcommand{\familydefault}{\sfdefault}

% This is a helpful package that puts math inside length specifications
\usepackage{calc}

% This package helps LaTeX auto-hyphenate hyphenated words if you use
% special hyphens. For example, bio\-/mimicry will properly hyphenate
% ``mimicry'' if necessary.
\usepackage[shortcuts]{extdash}

% Layout: Puts the section titles on left side of page
\reversemarginpar

%
%         PAPER SIZE, PAGE NUMBER, AND DOCUMENT LAYOUT NOTES:
%
% The next \usepackage line changes the layout for CV style section
% headings as marginal notes. It also sets up the paper size as either
% letter or A4. By default, letter was used. If A4 paper is desired,
% comment out the letterpaper lines and uncomment the a4paper lines.
%
% As you can see, the margin widths and section title widths can be
% easily adjusted.
%
% ALSO: Notice that the includefoot option can be commented OUT in order
% to put the PAGE NUMBER *IN* the bottom margin. This will make the
% effective text area larger.
%
% IF YOU WISH TO REMOVE THE ``of LASTPAGE'' next to each page number,
% see the note about the +LP and -LP lines below. Comment out the +LP
% and uncomment the -LP.
%
% IF YOU WISH TO REMOVE PAGE NUMBERS, be sure that the includefoot line
% is uncommented and ALSO uncomment the \pagestyle{empty} a few lines
% below.
%

%% Use these lines for letter-sized paper
\usepackage[paper=letterpaper,
            %includefoot, % Uncomment to put page number above margin
            marginparwidth=1.2in,     % Length of section titles
            marginparsep=.05in,       % Space between titles and text
            margin=1in,               % 1 inch margins
            includemp]{geometry}

%% Use these lines for A4-sized paper
%\usepackage[paper=a4paper,
%            %includefoot, % Uncomment to put page number above margin
%            marginparwidth=30.5mm,    % Length of section titles
%            marginparsep=1.5mm,       % Space between titles and text
%            margin=25mm,              % 25mm margins
%            includemp]{geometry}

%% More layout: Get rid of indenting throughout entire document
\setlength{\parindent}{0in}

% Provides special list environments and macros to create new ones
\usepackage[shortlabels]{enumitem}

% Simpler bibsections for CV sections
% (thanks to natbib for inspiration)
%
% * For lists of references with hanging indents and no numbers:
%
%   \begin{bibsection}
%       \item ...
%   \end{bibsection}
%
% * For numbered lists of references (with hanging indents):
%
%   \begin{bibenum}
%       \item ...
%   \end{bibenum}
%
%   Note that bibenum numbers continuously throughout. To reset the
%   counter, use
%
%   \restartlist{bibenum}
%
%   at the place where you want the numbering to reset.

\makeatletter
\newlength{\bibhang}
\setlength{\bibhang}{1em}
\newlength{\bibsep}
 {\@listi \global\bibsep\itemsep \global\advance\bibsep by\parsep}
\newlist{bibsection}{itemize}{3}
\setlist[bibsection]{label=,leftmargin=\bibhang,%
        itemindent=-\bibhang,
        itemsep=\bibsep,parsep=\z@,partopsep=0pt,
        topsep=0pt}
\newlist{bibenum}{enumerate}{3}
\setlist[bibenum]{label=[\arabic*],resume,leftmargin={\bibhang+\widthof{[999]}},%
        itemindent=-\bibhang,
        itemsep=\bibsep,parsep=\z@,partopsep=0pt,
        topsep=0pt}
\let\oldendbibenum\endbibenum
\def\endbibenum{\oldendbibenum\vspace{-.6\baselineskip}}
\let\oldendbibsection\endbibsection
\def\endbibsection{\oldendbibsection\vspace{-.6\baselineskip}}
\makeatother

%% Reference the last page in the page number
%
% NOTE: comment the +LP line and uncomment the -LP line to have page
%       numbers without the ``of ##'' last page reference)
%
% NOTE: uncomment the \pagestyle{empty} line to get rid of all page
%       numbers (make sure includefoot is commented out above)
%
\usepackage{fancyhdr,lastpage}
\pagestyle{fancy}
%\pagestyle{empty}      % Uncomment this to get rid of page numbers
\fancyhf{}\renewcommand{\headrulewidth}{0pt}
\fancyfootoffset{\marginparsep+\marginparwidth}
\newlength{\footpageshift}
\setlength{\footpageshift}
          {0.5\textwidth+0.5\marginparsep+0.5\marginparwidth-2in}
\lfoot{\hspace{\footpageshift}%
       \parbox{4in}{\, \hfill %
                    \arabic{page} of \protect\pageref*{LastPage} % +LP
%                    \arabic{page}                               % -LP
                    \hfill \,}}

% Finally, give us PDF bookmarks
\usepackage{color,hyperref}
\definecolor{darkblue}{rgb}{0.0,0.0,0.3}
\hypersetup{colorlinks,breaklinks,
            linkcolor=darkblue,urlcolor=darkblue,
            anchorcolor=darkblue,citecolor=darkblue}

%%%%%%%%%%%%%%%%%%%%%%%% End Document Setup %%%%%%%%%%%%%%%%%%%%%%%%%%%%


%%%%%%%%%%%%%%%%%%%%%%%%%%% Helper Commands %%%%%%%%%%%%%%%%%%%%%%%%%%%%

%%% HEADING AT TOP OF CURRICULUM VITAE

% The title (name) with a horizontal rule under it
% (optional argument typesets an object right-justified across from name
%  as well)
%
% Usage: \makeheading{name}
%        OR
%        \makeheading[right_object]{name}
%
% Place at top of document. It should be the first thing.
% If ``right_object'' is provided in the square-braced optional
% argument, it will be right justified on the same line as ``name'' at
% the top of the CV. For example:
%
%       \makeheading[\emph{Curriculum vitae}]{Your Name}
%
% will put an emphasized ``Curriculum vitae'' at the top of the document
% as a title. Likewise, a picture could be included:
%
%   \makeheading[{\includegraphics[height=1.5in]{my_picture}}]{Your Name}
%
% the picture will be flush right across from the name. For this example
% to work, make sure the extra set of curly braces is included. Also
% makes ure that \usepackage{graphicx} is somewhere in the preamble.
\newcommand{\makeheading}[2][]%
        {\hspace*{-\marginparsep minus \marginparwidth}%
         \begin{minipage}[t]{\textwidth+\marginparwidth+\marginparsep}%
             {\large \bfseries #2 \hfill #1}\\[-0.15\baselineskip]%
                 \rule{\columnwidth}{1pt}%
         \end{minipage}}

%%% SECTION HEADINGS

% The section headings. Flush left in small caps down pseudo-margin.
%
% Usage: \section{section name}
\renewcommand{\section}[1]{\pagebreak[3]%
    \vspace{1.3\baselineskip}%
    \phantomsection\addcontentsline{toc}{section}{#1}%
    \noindent\llap{\scshape\smash{\parbox[t]{\marginparwidth}{\hyphenpenalty=10000\raggedright #1}}}%
    \vspace{-\baselineskip}\par}

%%% LISTS

% This macro alters a list by removing some of the space that follows the list
% (is used by lists below)
\newcommand*\fixendlist[1]{%
    \expandafter\let\csname preFixEndListend#1\expandafter\endcsname\csname end#1\endcsname
    \expandafter\def\csname end#1\endcsname{\csname preFixEndListend#1\endcsname\vspace{-0.6\baselineskip}}}

% These macros help ensure that items in outer-type lists do not get
% separated from the next line by a page break
% (they are used by lists below)
\let\originalItem\item
\newcommand*\fixouterlist[1]{%
    \expandafter\let\csname preFixOuterList#1\expandafter\endcsname\csname #1\endcsname
    \expandafter\def\csname #1\endcsname{\let\oldItem\item\def\item{\pagebreak[2]\oldItem}\csname preFixOuterList#1\endcsname}
    \expandafter\let\csname preFixOuterListend#1\expandafter\endcsname\csname end#1\endcsname
    \expandafter\def\csname end#1\endcsname{\let\item\oldItem\csname preFixOuterListend#1\endcsname}}
\newcommand*\fixinnerlist[1]{%
    \expandafter\let\csname preFixInnerList#1\expandafter\endcsname\csname #1\endcsname
    \expandafter\def\csname #1\endcsname{\let\oldItem\item\let\item\originalItem\csname preFixInnerList#1\endcsname}
    \expandafter\let\csname preFixInnerListend#1\expandafter\endcsname\csname end#1\endcsname
    \expandafter\def\csname end#1\endcsname{\csname preFixInnerListend#1\endcsname\let\item\oldItem}}

% An itemize-style list with lots of space between items
%
% Usage:
%   \begin{outerlist}
%       \item ...    % (or \item[] for no bullet)
%   \end{outerlist}
\newlist{outerlist}{itemize}{3}
    \setlist[outerlist]{label=\enskip\textbullet,leftmargin=*}
    \fixendlist{outerlist}
    \fixouterlist{outerlist}

% An environment IDENTICAL to outerlist that has better pre-list spacing
% when used as the first thing in a \section
%
% Usage:
%   \begin{lonelist}
%       \item ...    % (or \item[] for no bullet)
%   \end{lonelist}
\newlist{lonelist}{itemize}{3}
    \setlist[lonelist]{label=\enskip\textbullet,leftmargin=*,partopsep=0pt,topsep=0pt}
    \fixendlist{lonelist}
    \fixouterlist{lonelist}

% An itemize-style list with little space between items
%
% Usage:
%   \begin{innerlist}
%       \item ...    % (or \item[] for no bullet)
%   \end{innerlist}
\newlist{innerlist}{itemize}{3}
    \setlist[innerlist]{label=\enskip\textbullet,leftmargin=*,parsep=0pt,itemsep=0pt,topsep=0pt,partopsep=0pt}
    \fixinnerlist{innerlist}

% An environment IDENTICAL to innerlist that has better pre-list spacing
% when used as the first thing in a \section
%
% Usage:
%   \begin{loneinnerlist}
%       \item ...    % (or \item[] for no bullet)
%   \end{loneinnerlist}
\newlist{loneinnerlist}{itemize}{3}
    \setlist[loneinnerlist]{label=\enskip\textbullet,leftmargin=*,parsep=0pt,itemsep=0pt,topsep=0pt,partopsep=0pt}
    \fixendlist{loneinnerlist}
    \fixinnerlist{loneinnerlist}

%%% EXTRA SPACE

% To add some paragraph space between lines.
% This also tells LaTeX to preferably break a page on one of these gaps
% if there is a needed pagebreak nearby.
\newcommand{\blankline}{\quad\pagebreak[3]}
\newcommand{\halfblankline}{\quad\vspace{-0.5\baselineskip}\pagebreak[3]}

%%% FORMATTING MACROS

% Provides a linked \doi{#1} that links doi:#1 to http://dx.doi.org/#1
\usepackage{doi}
% To change the text before the DOI, adjust this command
%\renewcommand\doitext{doi:}

% Provides a linked \url{#1} that doesn't require escape characters
\usepackage{url}

% You can adjust the style \url{} uses here:
% (options are: same, rm, sf, tt; defaults to tt)
\urlstyle{same}

% For \email{ADDRESS}, links ADDRESS to the url mailto:ADDRESS
% (uncomment to typeset the e\-/mail address in typewriter font;
%  otherwise, will be typeset in the \urlstyle above)
%\DeclareUrlCommand\emaillink{\urlstyle{tt}}
\providecommand*\emaillink[1]{\nolinkurl{#1}}
\providecommand*\email[1]{\href{mailto:#1}{\emaillink{#1}}}

\providecommand\BibTeX{{B\kern-.05em{\sc i\kern-.025em b}\kern-.08em \TeX}}
\providecommand\Matlab{\textsc{Matlab}}
\providecommand\Python{\textsc{Python}}

% Custom hyphenation rules for words that LaTeX has trouble with
\hyphenation{bio-mim-ic-ry bio-in-spi-ra-tion re-us-a-ble pro-vid-er Media-Wiki}

%%%%%%%%%%%%%%%%%%%%%%%% End Helper Commands %%%%%%%%%%%%%%%%%%%%%%%%%%%

%%%%%%%%%%%%%%%%%%%%%%%%% Begin CV Document %%%%%%%%%%%%%%%%%%%%%%%%%%%%

\begin{document}
\makeheading{Dr. Mahmut Pekedis}

 \section{Contact Information}

% NOTE: Mind where the & separators and \\ breaks are in the following
%       table. Table is one row made up of three parboxes. The left
%       parbox has address info, the middle parbox has a vertical bar,
%       and the right parbox has phone and electronic contact
%       information.
%
% MACROS: \rcollength is the width of the right column of the table
%             (adjust it to your liking; default is 1.85in).
%         \spacewidth is width of area between left and right boxes.
%
\newlength{\rcollength}\setlength{\rcollength}{1.85in}%
\newlength{\spacewidth}\setlength{\spacewidth}{20pt}
%
\begin{tabular}[t]{@{}p{\textwidth-\rcollength-\spacewidth}@{}p{\spacewidth}@{}p{\rcollength}}%

% Address box
\parbox{\textwidth-\rcollength-\spacewidth}{%
%Assistant Professor\\
\href{http://www.ege.edu.tr/}{Ege University}\\
\href{https://me.ege.edu.tr/eng-/Homepage.html}{Department of Mechanical Engineering}\\
35040, Bornova,\\Izmir, Turkey \\}

&
% Uncomment to add a vertical bar in middle of contact information
%{\vrule width 0.5pt}
\parbox[m][5\baselineskip]{\spacewidth}{} &

% Non-snail-mail contact information
\parbox{\rcollength}{%
\textit{Work:}   ~~+90-232-311-5122 \\
\email{mahmut.pekedis@ege.edu.tr}\\
}
%\textit{WWW:} \href{http://www.tedpavlic.com/}{www.tedpavlic.com}

\\

\end{tabular}

%%
%% In modern CV's, it seems like ``Objective'' is frowned upon. Instead,
%% incorporate it into a well-constructed cover letter. The ``More
%% information'' can go at the end of the CV, but it should not distract
%% from the section giving references available to contact.
%%
%
% \section{Objective}
%
% Placement in an academic position (i.e., faculty, postdoctoral, or
% research scientist) that allows for advanced research in distributed
% complex adaptive systems (i.e., modeling, analysis, design, and
% verification) with a particular focus on the control of engineered
% agents (e.g., for communications, control, software, electronics, and
% sustainability) and the analysis of biological phenomena (e.g.,
% self-organization, ecological rationality)
% \begin{innerlist}
% \item More information and auxiliary documents can be found at\\\url{http://www.tedpavlic.com/facjobsearch/}
% \end{innerlist}

\section{Research Interests}

\textbf{Structural mechanics:} structural health monitoring, biomechanics, numerical methods, the finite element analysis, meshless methods, damage detection, mechanics of composite materials, impact-crash analysis, in-silico modelling, elasticity theory, constitutive modelling, uncertainity quantification, validation and verification, clinical--experimental and numerical biomechanics, human-machine haptics interface, pattern recognation, machine learning, cyber physical systems, autonomous systems, robotics


\section{Current Academic Appointments}

\textbf{Associate Professor},
            \href{http://www.ege.edu.tr/}{Ege University}
            \hfill {August 2022 to present}
\begin{innerlist}

    \item[] \href{https://me.ege.edu.tr/eng-/Homepage.html}{Department of
    Mechanical Engineering}~(80\%)
        
    \item[] \href{https://egeplm.ege.edu.tr/eng-/Homepage.html}{Product 
    Lifecycle Management Research and Application Excellence Center}~(10\%)
        
    \item[] \href{https://ebp.ege.edu.tr/DereceProgramlari/Detay/2/61130/8230/932001?lang=en-US}{
    Mechatronics Engineering}~(10\%)

\end{innerlist}

\section{Previous Academic Appointments}

\textbf{Assistant Professor},
            \href{http://www.ege.edu.tr}{Ege University}
            \hfill {April 2015 to August 2021}
\begin{innerlist}
    \item[] \href{https://me.ege.edu.tr/eng-/Homepage.html}{Department of
    Mechanical Engineering}
\end{innerlist}


\halfblankline

\textbf{PhD Research Scholar},
            \href{https://www.lanl.gov}{Los Alamos National Laboratory}
            \hfill {June 2012 to August 2013}
\begin{innerlist}

    \item[] \href{https://www.lanl.gov/projects/national-security-education-center/engineering/}{Engineering
    Institute}
    \begin{innerlist}
        \item Supervisor: Dr. Charles Farrar and Dr. David Mascarenas
        \item A vibro-haptics human-machine interface for structural health monitoring
    \end{innerlist}
\end{innerlist}



\textbf{Research/Teaching Assistant},
            \href{http://www.ege.edu.tr/}{Ege University}
            \hfill {September 2009 to June 2012}
  \begin{innerlist}
    
        \item[] \href{https://me.ege.edu.tr/eng-/Homepage.html}{Department of
    Mechanical Engineering}    
    \begin{innerlist}

        \item Supervisor: \href{https://unisis.ege.edu.tr/researcher=hasan.yildiz}{Professor Hasan Yildiz}
        \item Structural health monitoring for composite structures
     \end{innerlist}
  \end{innerlist}




\section{Education}


\begin{innerlist}

\item[] Ph.D.,
        \href{https://me.ege.edu.tr/eng-/Homepage.html}
             {Mechanical Engineering},
             \href{https://www.ege.edu.tr}{Ege University}, September 2014
        \begin{innerlist}
        \item Thesis Topic: \emph{Structural Health Monitoring and 
        Maintanence System for Composite Materials}
        \item Adviser:
              \href{https://unisis.ege.edu.tr/researcher=hasan.yildiz}
                   {Professor Hasan Yildiz}
        \item Area of Study: Mechanical Engineering
        \end{innerlist}

\item[] M.S.,
        \href{https://me.ege.edu.tr/eng-/Homepage.html}
             {Mechanical Engineering},
             \href{https://www.ege.edu.tr}{Ege University}, August 2008
        \begin{innerlist}
        \item Thesis Topic: \emph{Structural Analysis with Meshfree Methods}
        \item Adviser:
              \href{https://unisis.ege.edu.tr/researcher=hasan.yildiz}
                   {Associate Professor Hasan Yildiz}
        \item Area of Study: Mechanical Engineering
        \end{innerlist}

\item[] B.S.,
        \href{https://www.gazi.edu.tr/}
             {Mechanical Engineering}, 
             \href{https://www.gazi.edu.tr}{Gazi University}, January 2006
\end{innerlist}

% \section{Submitted Journal Publications}
%
% % Add a little space to nudge next ``Ref'd Journal Publications'' marginpar
% % down to make room for tall ``Submitted Journal Publications''
% % marginpar. If there are enough submitted journal publications, this
% % space will not be needed (and should be removed).
% \vspace{0.1in}

\section{Refereed Journal Publications}

\begin{bibenum}

	\item	Ercan, E., Avcı, M.S.,Pekedis, M., Hızal, Ç. Damage classification of a three-story 
		aluminum building model by convolutional neural networks and the effect of scarce 		
		accelerometers.
		\emph{Appl. Sci.}, 14:2568, 2024.
		\doi{10.3390/app14062628}


	\item	Pekedis M., Ozan F., Koyuncu S., Yildiz H. The finite element method-based pattern recognition approach for
		the classification of patient-specific gunshot injury.
		\emph{Proceedings of the Institution of Mechanical Engineers, Part H: Journal of Engineering in Medicine},
		236(5):665--675, 2022.
		\doi{10.1177/09544119221086397}
		
	\item Pekedis M., Altan M., Akgul T., Yildiz H. The influence of accessory rods and connectors on the quasi-static 
		and dynamic response of spine fixation.\\ 
		\emph{Experimental Techniques}, 2022.
		\doi{10.1007/S40799-022-00569-2} 
	
	\item Özden M.A., Acar E., Yildiz H., Güner M., Pekedis M.  A vibro-haptics smart corset trainer for non-ideal 
		sitting posture.
		\emph{Textile and Apparel}, 32(4), 304 - 313, 2022.
		\doi{10.32710/tekstilvekonfeksiyon.994444}

	\item Pekedis M., Ozan F., Yildiz H. Biomechanics of the femoral head cartilage and subchondral 
		trabecular bone in osteoporotic and osteopenic fractures. 
		\emph{Annals of Biomedical Engineering}, 49(12), 3388-3400, 2021. 
		\doi{10.1007/s10439-021-02861-5}
		
	\item Pekedis M. Detection of multiple bolt loosening via data based statistical pattern recognition techniques.
		\emph{Journal of the Faculty of Engineering and Architecture of Gazi University}, 36(4), 1993-2010, 2021.
		\doi{10.17341/gazimmfd.820157}
		
	\item Budak I.N., Pekedis M. Experimental and numerical fatigue behaviour analysis of a plastically deformed 
		automobile tie rod.
		\emph{Dokuz Eylul University Engineering Faculty Journal of Science and Engineering}, 23(68), 647-659, 2021.\\
		\doi{10.21205/deufmd.2021236826}  

	\item Pekedis M., Yoruk M.D., Binboga E., Yildiz H., Bilge O., Çelik S. Characterization of the mechanical properties
		of human parietal bones preserved in modified larssen solution, formalin and as fresh frozen. 
		\emph{Surgical and Radiologic Anatomy}, 43(12), 1933-1943, 2021.
		\doi{10.1007/s00276-021-02762-1}
		
	\item Kabacaoglu S., Pekedis M., Yildiz H., Theoretical and experimental improvement of the effect of the diaphragm
		 springs form used in the clutch system on the fatigue strength and mechanical characteristics. 
		\emph{Uludag University Journal of the Faculty of Engineering}, 26(3), 1121--1138, 2021.
		\doi{10.17482/uumfd.939663} 		
		
	\item Ayran E., Pekedis M. Validation of experimental impact tests on aluminum alloy car wheels using the 
		finite element method. 
		\emph{DUJE}, 11(2), 663--670, 2020.\\ 
	 	\doi{10.24012/dumf.651318}
	 	 
	\item Pekedis M. Damage diagnosis of bolt loosening via vector autoregressive - support vector machines.i
		\emph{Hittite Journal of Science & Engineering}, 7(3), 169--179, 2020.
		\\\doi{10.17350/HJSE19030000186} 

	\item Ozan F., Pekedis M., Koyuncu S., Altay T., Yildiz H., Kayali C. Micro-computed tomography and mechanical
		evaluation of trabecular bone structure in osteopenic and osteoporotic fractures. 
		\emph{Journal of Orthopaedic Surgery}, 25(1), 1--6, 2017.
		\doi{10.1177/2309499017692718}
		
	\item Pekedis M., Yildiz H. Damage diagnosis of a laminated composite beam and plate via model based structural 
		health monitoring techniques. 
		\emph{Journal of the Faculty of Engineering and Architecture of Gazi University}, 31(4), 813--831, 2016.\\
		\href{https://dergipark.org.tr/en/download/article-file/259809} {link}
	
	\item Akdemir Ovunc, C Lineaweaver W., Cavusoglu T., Binboğa E., Uyanikgil Y., Zhang F., Pekedis M., Yagci T. Effect of taurine on rat Achilles tendon healing.
		 \emph{Connective Tissue Research}, 56(4), 300--306, 2015.\\ 
		 \doi{10.3109/03008207.2015.1026437}
		 
	\item Pekedis M., Masceranas D., Turan G., Ercan E., Farrar C.R, Yildiz H. Structural health monitoring for bolt
		 loosening via a non invasive vibro haptics human machine cooperative interface. 
		 \emph{Smart Materials and Structures}, 24(8), 85018, 2015.
		 \doi{10.1088/0964-1726/24/8/085018}
		 
	\item Pekedi̇s M., Yildiz H. Numerical analysis of a projectile penetration
		into the human head via meshless method.
		\emph{Journal of Mechanics in Medicine and Biology}, 14(04), 1450059, 2014.
		\doi{10.1142/S0219519414500596}
		
	\item Ozan F., Koyuncu S., Pekedis M., Altay T., Yildiz H., Toker G. Greater trochanteric
		fixation using a cable syste for partial hip arthroplasty: 
		A clinical and finite element analysis. 
		\emph{BioMed Research International}, 1--7, 2014.\\
		\doi{10.1155/2014/931537}
		
	\item Olmez S., Dogan S., Pekedis M., Yildiz H. Biomechanical evaluation of sagittal maxillary internal distraction osteogenesis
		 in unilateral cleft lip and palate patient and noncleft patients: A three-dimensional finite element analysis.
		 \emph{The Angle Orthodontist}, 84(5), 815--824, 2014.
		 \doi{10.2319/080613-586.1}
		 
	\item Atesci Y.Z., Aydogdu O., Karakose A., Pekedis M., Karal O., Senturk U., Cinar M. Does urinary bladder shape 
		affecturinary flow rate in men with lower urinary tract symptoms?. 
		\emph{The Scientific World Journal}, 1--5, 2014.\\
		\doi{10.1155/2014/846856} 
	
	\item Kuran F.D., Pekedis M., Yildiz H. ,Aydin F., Eliyatkin N. 	Effect of hyperbaric oxygen treatment on
		tendon healing after Achilles tendon repair: an experimental study on rats. 
		\emph{Acta Orthopaedica et Traumatologica Turcica}, 46(4), 293--300, 2012.
		\doi{10.3944/AOTT.2012.2653}
		
	\item Pekedis M., Yildiz H. Comparison of fatigue behaviour of eight different hip stems: a numerical and experimental study. 
		\emph{Journal of Biomedical Science and Engineering}, 4(10), 643--650, 2011.
		\doi{10.4236/jbise.2011.410080} 

	\item Atesci Y.Z, Senturk U., Pekedis M., Cinar M. Non-invasive urodynamic analysis using the computational fluid dynamics 
		method based on MR Images. 
		\emph{Turkiye Klinikleri Journal of Medical Sciences}, 31(5), 1186--1193, 2011.\\
		\doi{10.5336/medsci.2010-21420}
	
	\item Yegengil C., Pekeds M., Yildiz H. Fracture healing in a denervation and or nerve ending interpositioning model in the rat.
		 \emph{International Journal of Clinical Medicine}, 2(3), 301--306, 2011.
		 \doi{10.4236/ijcm.2011.23051}

	\item Pekedis M., Yildiz H. Meshfree Methods and Their Classification. 
		\emph{Pamukkale University Journal of Engineering Sciences}, 2010, 16(1), 1--9.
	
	\item Ozan F., Yildiz H., Bora O.A, Pekedis M., Ay Coskun G., Gore O. The Effect Of Head Trauma On Fracture Healing: Biomechanical
	 	Testing And Finite Element Analysis. 
	 	\emph{Acta Orthopaedica Et Traumatologica Turcica}, 44(4), 313-321, 2010.\\
	 	\Doi{10.3944/Aott.2010.2277}
	 	
	 \item Pekedis M., Yildiz H. Modelling of axially loaded cantilever rod using element free Galerkin method. 
	 	\emph{Pamukkale University Journal of Engineering Sciences}, 15(3), 353--361, 2009.
	 
	 \item Pekedis M., Yildiz H. Solution of 2D cantilever beam by using the element free Galerkin method with the 
	 	finite element method. 
	 	\emph{Sigma Journal of Engineering and Natural Sciences}, 27(1), 26--38, 2009. 
	
	
\end{bibenum}

% Add a little space to nudge next ``Conference Publications'' marginpar
% down to make room for tall ``Submitted Conference Publications''
% marginpar. If there are enough submitted journal publications, this
% space will not be needed (and should be removed).
\vspace{0.1in}

\section{Refereed Conference Publications}

\begin{bibenum}

	\item Avci M.S., Ercan E., Nuhoğlu A., Arisoy B., Hizal Ç., Pekedis M. Evaluating Damage 		Detection Performance in Reinforced Concrete Columns Using Synthetic Data and Machine 			Learning
	\emph{5th International Engineering Research Symposium}, March 7-92024, Duzce, Turkey.
		

	\item Avci M.S.*, Ercan E., Nuhoğlu A., Arisoy B., Hizal Ç., Pekedis M., Structural Damage 		Detection on Fire-Exposed Reinforced Concrete Columns Using Deep Learning: A Study with 		Acceleration Data
	\emph{5th International Engineering Research Symposium}, March 7-92024, Duzce, Turkey.


	\item Pekedis M. Uncertainity quantification of hyperelastic soft tissue.  
		\emph{26th Congress of the European Society of Biomechanics}, pp.~612, July 11--14, 2021. Milan, Italy.

	\item Pekedis M. Using Bayesian framework to calibrate Voce model parameters of ductile human parietal bone. 
		\emph{Biomechanics 2020}, pp.~119-120, September 9--10, 2021. Warsaw, Poland.

	\item Pekedis M. Biomechanical comparison of the osteoporotic and osteopenic trabecular bone specimens using 
		modal analysis, 2021.
		\emph{2nd International Congress on Engineering Sciences and Multidisciplinary Approaches}, pp.~325, 
		September 18, 2021. Istanbul, Turkey.
	
	\item Yörük M.D., Pekedis M., Binboğa E., Çelik S., Bilge O. A comparison of biomechanical features modified Larssen 
		fixed, 10\% formalin fixed and fresh frozen cadavers.
		\emph{National Anatomy Congress & 1st International Mediterranean Anatomy Congress}, 12(2),
		September 6--9, 2018. Konya, Turkey.

		
	\item Pekedis M., Ozan F., Koyuncu Ş., Yildiz H., Application of the numerical method with machine learning algorithm 
		for a 3D specific firearm injury forensic model. 
		\emph{23rd Congress of the European Society of Biomechanics}, July 2--5, 2017. Seville, Spain. 
		
	\item Pekedis M., Masceranas D., Turan G., Ceylan H., Ercan E., Farrar C.R, Yildiz H. Structural health monitoring via 
		human-machine Interface. \emph{Sixth World Conference on Structural Control and Monitoring},
		pp.~167--171, July 15--17, 2014. Barcelona, Spain.

	\item Ercan E., Pekedis M., Turan G., Ceylan H. Structural health monitoring with audio presentations 
		\emph{Sixth World Conference on Structural Control and Monitoring}, pp.~2643--2649, July 15--17, 2014. Barcelona, Spain.

	\item Ceylan H., Turan G., Ercan E., Pekedis M.	Structural damage detection by using a single 
		excitation record.	\emph{Sixth World Conference on Structural Control and Monitoring},
		pp.~2567--2574, July 15--17, 2014. Barcelona, Spain.
		
	\item Mascarenas D.D.L., Choi YS., Kim H.C, Pekedis M., Hong S.C, Lee J.R, Farrar C.R. Development of 
		a novel human-machine interface exploiting sensor substitution for structural health monitoring. 
		\emph{2013 IEEE RO-MAN: The 22nd IEEE International Symposium on Robot and Human Interactive Communication}, 
		August 26--29, 2013. Gyeongju, South Korea.  
		\doi{10.1109/ROMAN.2013.6628482}. 
			
	\item Pekedis M., Yildiz H. Simulation of a projectile penetration to human head via meshless method.
		\emph{The 11th International Symposium on Computer Methods in Biomechanics and Biomedical Engineering}, 
		April 3--4, 2013. Salt Lake City, Utah, USA. 
		
	\item Mascarenas D., Choi Y.S, Kim H.C, Pekedis M., Yildiz H., Plont C., Brown C, Cowell M, Park G, Hahn H, 
		Lee Jung-Ryul, Farrar C.R. A vibro-haptic human-machine interface for structural health monitoring. 
		\emph{9th International Workshop on Structural Health Monitoring (IWSHM)}, 2, pp.~1171-1178, September 10--12, 2013.
		San Francisco, CA, USA.
	
	\item Pekedis M., Yildiz H. Non-destructive damage detection of laminated composite beams based on dynamic analysis techniques.
		\emph{9th International Fracture Conference}, pp.~114--124, September 19--21, 2011. Istanbul, Turkey.
		
	\item Pekedis M., Yildiz H., Comparison of dynamic fatigue behavior of eight different implanted hip prostheses during gait.
		\emph{9th International Fracture Conference}, pp.~397--407, September 19--21, 2011. Istanbul, Turkey. 

	\item Pekedis M., Ozan F., Yildiz H. Developing of a three-dimensional foot ankle model based on CT images and non-linear 
		analysis of anterior drawer test by using the finite element method. 
		\emph{Fifth International Participated National Biomechanics Congress}, 44(1), pp.~14, September 23--25, 2010. Izmir, Turkey. 
		\\ \doi{10.1016/j.jbiomech.2011.02.054}
		
	\item Duran D., Kaya E., Pekedis M., Şentürk U., Erkek M., Yildiz H. Applications of numerical methods in biomechanics. 
		\emph{Fifth International Participated National Biomechanics Congress}, 44(1), pp.~14-15, September 23--25, 2010. Izmir, Turkey.
		\doi{10.1016/j.jbiomech.2011.02.055}


\end{bibenum}

\vspace{0.1in}

\section{Conference Posters}

\begin{bibenum}

	\item Güner M., Yildiz H., Pekedis M., Acar E., Şekeroğlu S., Özden M.A.
		Improving a smart corset for posture defects.
		\emph{Mas International Conference on Mathematics, Engineering, Natural & Medical Science-VI}, pp.~69, 
		July 12-14, 2019. Kiev, Ukraine. 

	\item Pekedis M., Ölmez S., Doğan S., Yildiz H. Patient-specific 3D surgical simulation of
		distraction osteogenesis and lefort I osteotomy using the finite element method.
		\emph{The 11th International Symposium on Computer Methods in Biomechanics and 
		Biomedical Engineering}, April 3--4, 2013. Salt Lake City, Utah, USA.2013.
		
	\item Ölmez S., Doğan S., Pekedis M., Yildiz H. The evaluation of maxillary advancement technique
		using internal distraction osteogenesis in unilateral cleft lip and palate patient with 
		finite element analysis.
		\emph{Fifth International Participated National Biomechanics Congress}, 44(1), 13-14, 
		September 23--25, 2010. Izmir, Turkey.
		\doi{10.1016/j.jbiomech.2011.02.053}
		
	\item Pekedis M., Yildiz H. Finite element analysis of the anterior drawer test for foot ankle.
		\emph{European Biotechnology Congress 2011}, 22(1), 150-151, September 28--October 1, 2011. Istanbul, Turkey.
		\emph{10.1016/j.copbio.2011.05.502},
		
	\item Ölmez S., Doğan S., Pekedis M., Yildiz H. Biomechanical analysis of sagittal maxillary advancement
		using internal maxillary distractors.
		\emph{87th Congress of the European Orthodontıc socıety}, June 19--23, 2011. Istanbul, Turkey.
		
		
	\item Ölmez S., Doğan S., Pekedis M., Yildiz H. Biomechanical effects of maxillary advancement on 
		the craniofacial skeleton with unilateral cleft lip and p	alate,
		using internal maxillary distractors.
		\emph{9 th European Craniofacial Congress}, September 14--17, 2011. Salzburg, Istanbul, Austria.		
		
	\item Pekedis M., Ozan F., Dinç M.H., Yildiz H. Investigation of the effects of allogeneic mesenchymal
		stem cells on bone union and regeneration in the necrosed bone by biomechanical test and the finite
		element method.
		\emph{Fifth International Participated National Biomechanics Congress}, 44(1), 15, September 23--25, 2010.
		Izmir, Turkey.
		\doi{10.1016/j.jbiomech.2011.02.056}


\end{bibenum}

\vspace{0.1in}
\section{Book Chapters}

\begin{bibenum}


	\item Ozdan MA., Pekedis M. Handbook of Aviation Technology and its Applications.
	In: C.~Harmansah, H.T.~Hava, 
	\emph{Real-time structural health monitoring using ultrasonic wave scatterers}. pp.~163--172, 2022.
	

\end{bibenum}
\vspace{0.1in}
\section{Other Publications}

\begin{bibenum}

    \item Pekedis M. \emph{Structural Health Monitoring and 
        Maintanence System for Composite Materials}. 
        PhD thesis, Ege University,
        Bornova, Izmir, Turkey, 2014.

    \item Pekedis M. \emph{Structural Analysis with Meshfree Methods}.
        Master's thesis, Ege University,
        Bornova, Izmir, Turkey, 2008
\end{bibenum}
\vspace{0.1in}
\section{Grants}
\restartlist{bibenum}

%\textbf{Awaiting Decision}
%
%\begin{bibenum}
%    \item Senior staff, ``A new multi-objective optimization framework
%        for investigating mechanisms of social resource allocation'',
%        NIH, NIGMS, 2015. Revision in progress.
%
%\end{bibenum}
%
%\blankline


\begin{bibenum}



	\item Pekedis M., Ozan F., Unlu ÖC ,Development of a graphical user interface based on 	
		coupled in vitro measurements and in silico modeling for the elastic-plastic biomechanical
		characterization of trabecular bone. August 16, 2023 to August 16, 2025.


     \item Hızal, H., Nuhoglu, A., Ercan E., Pekedis M., Avci MS Modal Parameter Determination
     Based on Probabilistic Distribution of Frequency Response Function.  Council of 
     	Higher Education -- Ege University Office of Scientific Research Projects,
     	February 27, 2024 to February 27, 2025.	


     \item Sarikanat M., İşbilir A., Gürses B.O, Uzunbayır B., Tümer D., Yildiz H., 
     	Yilmazkarasu H., Çetin L., Altay L., Pekedis M., Şener M., Boztepe M., 
     	Kaya T.Z, Seki Y. Development of Piezoelectric Polymeric Metamaterials for Space
     	and Satellite Technologies using Additive Manufacturing Technologies. Council of 
     	Higher Education -- Ege University Office of Scientific Research Projects,
     	February 1, 2023 to February 1, 2026.	
				
     																
	\item Ercan E.,  Nuhoğlu A., Arisoy B., Arisoy B.,  Hizal Ç., Pekedis  M., Avci M.S.
		Experimental and Numerical Evaluation of the Performance of Steel Composite Elements
		Embedded in Fired Concretes Subjected to Combined Flexural Loads. 
		Ege University Office of Scientific Research Projects, February 1, 2023 to February 1, 2024.		
		
	\item Yüce M.Ö, Koyuncu B.Ö., Tümer D., Gökçe F.M, Pekedis M.
		The Influences of Roots Left in the Mandible After Coronectomy on Mandible Fragility.
		Ege University Office of Scientific Research Projects, June 33, 2022 to June 24, 2024.
	
	\item Ozkol M.Z, Yildiz H., Biçer M.E.K, Pekedis M. Design, Manufacture and Anaerobic Performance 
		Testing of an Adjustable Q-Factor Cycling Prototype. The Scientific and Technological
		Research Council of Turkey (TUBITAK 1001), August 1, 2022 to August 1, 2024.
		
	\item  Yildiz H., Pekedis M., Pekbey Y., Unal H.Y. Improvement of Mechanical Properties
		of Composite Plates via Optimized Nanoclay Additions. Ege University Office 
		of Scientific Research Projects, June 29, 2017 to January 1, 2021.
		
	\item Pekedis M., Sayer S., Yildiz H. Development of a Human-Machine Haptics Interface 
		for Product Life-Cycle Management (PLM) Applications. Ege University Office of Scientific 
		Research Projects, June 29, 2017 to December 3, 2020.
							
	\item Yildiz H., Sayer S., Pekedis M.,Turan M., Unal H.Y, Gurses B.O. Optimization of a
		Laptop Chassis using Inverse Techniques for the Product Life-Cycle Management (PLM) Applications.
		Ege University Office of Scientific Research Projects. June 7, 2016 to January 1, 2019.
		

	\item	Guner M., Yildiz H., Pekedis M. Devolopment of a Smart Corset for Non-Ideal
		Sitting posture defects. The Scientific and Technological Research Council of Turkey (TUBITAK 1005), 
		January 15, 2018 to July 14, 2019.
		
 	\item Turan M., Ünal H.Y, Yildiz H., Pekedis M.,Sayer S. Improvement of Services in a Department
 		within the University: PLM Implementation. Ege University Office of Scientific Research Projects,
 		 February 27, 2018 to August 27, 2022.
 		 
	\item Yildiz H., Pekedis M. Modeling of the Delamination in Composite structures
		with Piezoelectric Materials. Ege University Office of Scientific Research Projects, 
		June 4, 2012 to June 18, 2015.
		
	\item Pekedis M, Yildiz H. A Human-machine Interface for Localizing Ultrasonic Scatterers
		for structural health monitoring. Ege University Office of Scientific Research Projects, 
		October 10, 2012 to December 12, 2013.
	
	\item Sarikanat M, Yildiz H., Baltaci A., Pekedis M, Erden S. Production, Improvement and
		Characterization of Natural Fiber Reinforced Polymer Composites. Ege University
		Office of Scientific Research Projects, April 20, 2009 to February, 2012.



\end{bibenum}

%\blankline
%
%\textbf{Not Awarded}
%
%\begin{bibenum}
%
%    \item Senior staff, ``Informational architecture of collective
%        decision making by \emph{Temnothorax} ants'', NSF, POLS, 2013.
%        Not awarded.
%
%    \item Senior staff,
%        ``Biological stoichiometry of horizontal gene transfer and the
%        social dynamics of microbial communities'', Army Research
%        Office, 2013. Not awarded.
%
%    \item Senior staff,
%        ``Biologically-inspired strategies for collective transport and
%        construction by multi-robot systems'', NSF, RI, 2013. Not
%        awarded.
%
%    \item Co\-/PI,
%        ``An Ant Model System for the Study of Nutrient Balance in
%        Social\-/Insect Pollinators'', USDA,
%        NIFA\-/AFRI Foundational proposal, 2013. Not awarded.
%
%    \item Senior staff,
%        ``Cooperative LED Arrays for Preference\-/Adaptive Lighting in
%        Smart Buildings'',
%        NSF,
%        EFRI\-/SEED preliminary proposal, 2009. Not awarded.
%
%\end{bibenum}

%\section{Academic Service}
%
%\textbf{Arizona State University, Tempe, AZ}
%\begin{lonelist}
%
%    \item College of Liberal Arts and Sciences Research Operations Committee,
%        \emph{Center for Social Dynamics and Complexity Representative},
%        2016--present.
%
%    \item Biosocial Complexity Initiative Directorate,
%        \emph{Liaison for Cross-University Activities},
%        2016--present.
%
%    \item Engineering Management Undergraduate Program Committee,
%        \emph{Member},
%        2015--present.
%
%    \item The Biomimicry Center,
%        \emph{Associate Director of Research},
%        2015--present.
%
%    \item Committee for the Development of Biomimicry and
%        Bio\-/inspired Research and Education Initiatives at ASU,
%        \emph{Chairman}.
%        2013.
%
%    \item Interdisciplinary Complexity Science Student Organization,
%        \emph{Founding faculty co\-/adviser}.
%        2013.
%
%\end{lonelist}

\section{Advising and Mentoring}

\textbf{Graduate Students}

\begin{outerlist}
%    \item \textbf{Brian Vincent}, PhD Student, Computer Science, 2017--\\
    \item \textbf{Mehmet Arda Özden}, PhD Student, Mechanical Engineering, 2020--
    \item \textbf{Güner Boyaci}, MS Student, Mechanical Engineerings, 2019--
    \item \textbf{Ali Düdük}, MS Student, Mechanical Engineering, 2018--
    \item \textbf{Mert Narin}, MS Student, Mechanical Engineering, 2020--
    \item \textbf{Tolga Aydin}, MS Student, Mechanical Engineering, 2021--     
    \item \textbf{Tuğba Kapan}, MS Student, Biomedical Technologies, 2020--
    \item \textbf{Schrin Schironova Sadriddin}, MS Student, Mechanical Engineering, 2021--          
    \item \textbf{Mert Mutlu Yilmaz}, MS Student, Mechanical Engineering, 2020--
    \item \textbf{Seda Hasanoğlu}, MS Student, Mechanical Engineering, 2020-- 
 
    \item \textbf{Saran Sapmaz}, MS Student, Mechatronics Engineering. \\
    	Thesis topic: Computer vision and sensor based autonomous mobile robot for 
    	simultaneous localization and mapping. 2020--2023 (graduated, MS) 	
   
    \item \textbf{Mehmet Arda Ozden}, MS Student, Mechanical Engineering. \\
    	Thesis topic: A human-machine auditory interface for structural
    	health monitoring 2018--2021 (graduated, MS)   	
    	
    \item \textbf{Emrah Ayran}, MS Student, Mechanical Engineering. \\
    	Thesis topic: Validation of experimental impact tests on aluminum alloy car
    	wheels using the finite element method. 2016--2019 (graduated, MS)
    	
    \item \textbf{Isin Naz Budak}, MS Student, Mechanical Engineering. \\
    	Thesis topic: Experimental and numerical fatigue behavior analysis of
    	a plastically deformed passenger car tie rod. 2016--2019 (graduated, MS)

\end{outerlist}

\vspace{0.1in}
\section{Teaching Experience}

\href{http://www.ege.edu.tr}{\textbf{Ege University}}, Izmir, Turkey
\begin{outerlist}

\item[] \textit{Instructor} \hfill \textbf{Fall 2015 to present}
    \begin{innerlist}
        \item Statics
        \item Dynamics
        \item Mechanics of Materials
        \item Computational Mechanics
        \item Elasticity Theory (Graduate-level course)
    \end{innerlist}



\item[] \textit{Instructor} \hfill \textbf{Fall 2019 to present}
    \begin{innerlist}
        \item Virtual product development principles (Graduate-level course)
    \end{innerlist}

\end{outerlist}

\section{Professional Service}

\textbf{Referee Service}
\begin{innerlist}
    \item \emph{Inverse Problems in Science & Engineering}
    \item \emph{Mechanical Systems and Signal Processing}
    \item \emph{Smart Materials and Structures}
    \item \emph{Prosthetics & Orthotics International}
    \item \emph{Computers in Biology and Medicine}
    \item \emph{Annals of Biomedical Engineering}
    \item \emph{Computer Methods in Biomechanics and Biomedical Engineering}
    \item \emph{Proceedings of the Institution of Mechanical Engineers, 
    	Part H: Journal of Engineering in Medicine}
    \item \emph{Journal of Tissue Engineering and Regenerative Medicine}
    \item \emph{Textile and Apparel}
    \item \emph{Frontiers in Psychiatry}
    \item \emph{Histology and Histopathology}
    \item \emph{Uludag University Journal of the Faculty of Engineering}
    \item \emph{Journal of Interdisciplinary Innovation Studies}
    \item \emph{Turkish journal of engineering & environmental sciences}
    \item \emph{Sigma Journal of Engineering and Natural Sciences}
    \item \emph{Acta Orthopaedica et Traumatologica Turcica}
    \item \emph{Nature Scientefic Reports}
    \item \emph{Frontiers in Surgery}
 
\end{innerlist}

\section{Professional Experience}

\href{http://www.ege.edu.tr}{\textbf{Ege University, Izmir, Turkey}}


    \item[] \textit{Associate Professor}%
            \hfill \textbf{August 2022}
                \item Joint Appointment:
                    \begin{innerlist}
                        \item \href{https://me.ege.edu.tr/eng-/Homepage.html}{Department of Mechanical Engineering}}
                        \item \href{https://egeplm.ege.edu.tr/eng-/Homepage.html}{Product Lifecycle Management
                        Research and Application Excellence Center}
                      \end{innerlist}
        
        
    \item[] \textit{Assistant Professor}%
            \hfill \textbf{April 2015 to August 2022}\\
            \href{https://me.ege.edu.tr/eng-/Homepage.html}{Department of Mechanical Engineering}
           \begin{innerlist}
           \item Devoloped a model based structural health monitoring
           		technique to detect the damage in composite materials
           
           \item Implemented data based structural health monitoring approaches
           	to detect nonlinearities in engineering systems
           
			\item Initiated a project for exploring the 
				use of ultrasonic-wave propagation measurements coupled 
				with non-invasive, vibro-tactors to develop a structural 
				health monitoring-human machine interface. 

			\item Performed a research to develop, test and validate
				a meshless smoothed particle hydrodynamic framework for the
				analysis of hard tissue failure.

			\item completed a research with orthopaedicians to investigate the 
				biomechanics of the femoral head cartilage and subchondral trabecular bone in
				osteoporotic and osteopenic fractures

            \item Chosen to deliver lecturers and prepare course 
                materials for undergraduate courses such as Elasticity theory,
                Mechanics of materials and Statics 
            
             \item Supervision of graduate and undergraduate students
                in mechanical, biomedical and mechatronics engineering.
            \end{innerlist}

 \item[] \textit{Research/Teaching Assistant}%
            \hfill \textbf{September 2009 to July 2012}\\
            \href{https://me.ege.edu.tr/eng-/Homepage.html}{Department of Mechanical Engineering}
         \item Supervisor: \href{https://unisis.ege.edu.tr/researcher=hasan.yildiz}{Professor Hasan Yildiz}                       
           \begin{innerlist}
                 
		\item Developed an algorithm based on modal shapes
			for detecting and repairing the crack damage in laminated
			composite beams.
		\item Support vector machines (SVM) a machine learning algorithm
		 has been applied to detect the presence of damage in a tower structure.
		\item Submitted the results of MS dissertation to scientific journals.
		\item Assisted some undergraduate courses such as the finite element method,
		mechanics of composite materials, strength of materials and statics.
            
 		\end{innerlist}
 		
  \halfblankline
 		
   \href{https://www.lanl.gov}{\textbf{Los Alamos National Laboratory, NM, USA}}
    \item[] \textit{PhD Research Scholar} \hfill \textbf{June 2012 to August 2013}\\
    \href{https://www.lanl.gov/projects/national-security-education-center/engineering/}{Engineering
    Institute}

    \item Supervisor: Dr. Charles Farrar and Dr. David Mascarenas
    
        \begin{innerlist}
                 
		\item Developed a novel human-machine cooperative paradigm
			for structural health monitoring applications. 
		\item The paradigm includes multidisciplinary fields such as human nervous system, 
			structural health monitoring (SHM) sensor networks, IoT networks, 
			signal processing techniques,  feature extractions methods, cyber-physical systems,
			 multivariate statistical methods, psychophysics procedures and h
			 aptics technology to create harmonies among human and machine.  
            
 		\end{innerlist}

  \halfblankline
 		
   \href{https://www.stfa.com/en/}{\textbf{STFA, Itanbul, Turkey}}
    \item[] \textit{Maintenance Engineer} \hfill \textbf{December 2007 to August 2008}
        \begin{innerlist}
                 
		\item Devoloped computer based fault diagnosis
			tools to determine the damages in construction equipments.
 		\end{innerlist}




\end{outerlist}

\section{Professional Memberships}

\href {https://esbiomech.org}{European Society of Biomechanics}, Member, 2017--present

\section{Other Meeting Attendance}

\textbf{Participant}

    \item Raising Awareness on Product Lifecycle Management via Education and 
    Industrial Strategic Collaboration within Europe, July 11 – July 15, 2016, Kaiserslautern, Germany.
    

\section{Service}

\href{https://me.ege.edu.tr/eng-/Homepage.html}{Ege University, 
Department of Mechanical Engineering}

	 \textit{Vice Chair}, \hfill \textbf{2021 to 2022}



\section{Application Areas}

Autonomous and Unmanned Vehicles, Mechanical Systems, Engineering Structures,
Flexible Manufacturing Systems, Biomedical Technologies

\section{Hardware and Software Skills}

Computer\-/Aided Design Tools:
%
\begin{innerlist}
    \item AutoCad, SolidWorks, Shapr3D, SpaceClaim, FreeCAD
\end{innerlist}

\halfblankline

Computer\-/Aided Engineering Tools:
%
\begin{innerlist}
    \item Abaqus/Implicit, Abaqus/Explicit, HyperWorks, HyperMesh, Ansys, CalculiX
\end{innerlist}

\halfblankline

Embedded and Real\-/time Systems:
%
\begin{innerlist}
    \item Software and hardware development with several MCU (e.g.,  Atmel
        ATmega MCU's, STM MCU's and others)
\end{innerlist}

\halfblankline

Instrumentation, Control, Data Acquisition, Test, and Measurement:
%
\begin{innerlist}
    \item \href{http://www.ni.com/}{LabVIEW} and other
        \href{http://www.ni.com}{National Instruments}
        control and data acquisition hardware and software (PicoScope, 
        Hewlett\-/Packard, Agilent, PI (Physik Instrumente), Keysight, 
        Tektronix equipments)
\end{innerlist}

\halfblankline

Computer Programming:
%
\begin{innerlist}
    \item C, Python, Fortran, Visual Basic (.NET) and others
\end{innerlist}

\halfblankline

Numerical Analysis:
%
\begin{innerlist}
    \item \Matlab, \Python~Scipy--NumPy
\end{innerlist}

\halfblankline



\href{http://www.mathworks.com/products/matlab/}{\Matlab} skill set:
%
\begin{innerlist}
    \item Linear algebra, symbolic math, Fourier transforms, Monte Carlo analysis,
        nonlinear numerical methods, polynomials, signal processing, visualization, machine learning

    \item Toolboxes: communications, control system, filter design, 
    statistics and machine learning, symbolic, econometrics
\end{innerlist}

\href{https://www.python.org}{\Python} skill set:
%
\begin{innerlist}
    \item Linear algebra, Symbolic math, Fourier transforms, Bayesian optimization, 
    Classification, Sensitivity Analysis, Nonlinear numerical methods, Signal processing, Machine learning,
     Visualization

    \item Modules: PyQt5, Qtgraph, Scipy, SymPy, NumPy,  Tensorflow, PyTorch, BoTorch,
    Scikit-optimize, Statsmodels, Seaborn, Arviz, Pandas and others
\end{innerlist}


\halfblankline

Information/Internet Technology:
%
\begin{innerlist}
    \item Networking (UDP, TCP), Data application
     and dashboard (Streamlit), Web framework (FastAPI)
\end{innerlist}

\halfblankline

Desktop Editing and Productivity Software:
%
\begin{innerlist}
    \item Eclipse, Jupyter, Spyder, PyCharm
    \item \TeX{} (\LaTeX{}),
    \item Microsoft Office, OpenOffice.org, LibreOffice, 
    \item InkScape
\end{innerlist}

\halfblankline

Operating Systems:
%
\begin{innerlist}
    \item Microsoft Windows family, Mac OS X, Ubuntu and other Linux variants
\end{innerlist}


\section{Awards}

\href{https://www.yok.gov.tr/en}{Council of Higher Education}

\begin{innerlist}
 \item Graduate Research Fellowship, 2012--2013
\end{innerlist}

\href{https://www.ege.edu.tr}{Ege University}
\begin{innerlist}
 \item Research Fellowship, 2012--2014
\end{innerlist}



\end{document}

% END


